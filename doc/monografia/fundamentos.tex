\chapter{Fundamentos}
\label{cap:fundamentos}

Neste capítulo, apresentamos algumas definições que o leitor deve conhecer para compreender o trabalho.

Partimos do pressuposto de que o leitor conhece notações básicas de conjuntos.

\begin{definition}[grafo]
  \cite{defgrafo}
  Um grafo é um par ordenado $G = (V, E)$. Os elementos de $V$ são chamados de vértices de $G$. Os elementos de $E$ são chamados de arestas de $G$ e consistem em pares (não-ordenados) de vértices. Dados $u, v \in V$, se $(u, v) \in E$ dizemos que $u$ e $v$ são adjacentes em $G$.
\end{definition}

\begin{definition}[grafo completo]
  \cite{defgrafocompleto}
  Um grafo $G = (V, E)$ é dito completo se $(u, v) \in E$ para todo $u, v \in V, u \neq v$.
\end{definition}

\begin{definition}[subgrafo induzido]
  \cite{defsubgrafo}
  Dado um grafo $G = (V, E)$ e um subconjunto $V'$ de $V$, o subgrafo de $G$ induzido por $V'$, $G' = (V', E')$, é o grafo formado pelos vértices $V' \subseteq V$ e arestas que só contém elementos de $V'$, ou seja, $E' = \{(u, v) \in E | u, v \in V'\}$.
\end{definition}

\begin{definition}[caminho]
  \cite{defcaminho}
  Dado um grafo $G = (V, E)$, um caminho é uma sequência de arestas que conectam uma sequência de vértices adjacentes distintos.
\end{definition}

\begin{definition}[distância]
  \cite{defdistancia}
  Dado um grafo $G = (V, E)$ e dois vértices $(u, v) \in V$, a distância entre $u$ e $v$ é o número de arestas num menor caminho que os conecte.
\end{definition}

\begin{definition}[árvore]
  \cite{defarvore}
  Dado um grafo $G = (V, E)$, dizemos que ele é uma árvore se cada dois vértices $u, v \in V$ são conectados por exatamente um caminho.
\end{definition}

\begin{definition}[$k$-clique]
  \label{def:kclique}
  \cite{defkclique} Seja $G = (V, E)$ um grafo. Um $k$-clique é um subconjunto dos vértices, $C \subseteq V$, tal que $(u, v) \in E \ \forall \ u, v \in C, u \neq v$ (ou seja, tal que o subgrafo induzido por $C$ é completo).
\end{definition}

\begin{definition}[$k$-tree e $k$-tree enraizada]
  \label{def:ktree}
  \cite{harary} Uma $k$-tree é definida da seguinte forma recursiva:

  \begin{enumerate}
    \item Um grafo induzido por um $k$-clique é uma $k$-tree.
    \item Se $T_k' = (V, E)$ é uma $k$-tree, $K \subseteq V$ é um $k$-clique e $v \not \in V$, então $T_k = (V \cup \{v\}, E \cup \{(v,x) \ | \  x \in K\})$ é uma $k$-tree.
  \end{enumerate}

  Uma $k$-tree enraizada é uma $k$-tree com um $k$-clique destacado $R = \{r_1, r_2, \cdots, r_k\}$ que é chamado de \emph{raiz} da $k$-tree enraizada.
\end{definition}

\begin{definition}[$k$-tree de Rényi]
  \cite{renyi} Uma $k$-tree de Rényi $R_k$ é uma $k$-tree enraizada com $n$ vértices rotulados em $[1, n]$ e raiz $R = \{n-k+1, n-k+2, \cdots, n\}$.
\end{definition}

\begin{definition}[esqueleto de uma $k$-tree enraizada]
  \label{def:skeleton}
  \cite{caminiti} O esqueleto de uma $k$-tree enraizada $T_k$ com raiz $R$, denotado por $S(T_k, R)$, é definido da seguinte forma recursiva:

  \begin{enumerate}
    \item Se $T_k$ é apenas o $k$-clique $R$, seu esqueleto é uma árvore com um único vértice $R$.
    \item Dada uma $k$-tree enraizada $T_k$ com raiz $R$, obtida por $T_k'$ enraizada em $R$ através da adição de um novo vértice $v$ conectado a um $k$-clique $K$ (ver definição \ref{def:ktree}), seu esqueleto $S(T_k, R)$ é obtido adicionando a $S(T_k', R)$ um novo vértice $X = \{v\} \cup K$ e uma nova aresta $(X, Y)$, onde $Y$ é o vértice de $S(T_k', R)$ que contém $K$ com uma distância mínima da raiz. Chamamos $Y$ de pai de $X$.
  \end{enumerate}
\end{definition}

\begin{definition}[árvore característica]
  \cite{caminiti} A árvore característica $T(T_k, R)$ de uma $k$-tree enraizada $T_k$ com raiz $R$ é obtida rotulando os vértices e arestas de $S(T_k, R)$ da seguinte forma:

  \begin{enumerate}
    \item O vértice $R$ é rotulado $0$ e cada vértice $\{v\} \cup K$ é rotulado $v$;
    \item Cada aresta do vértice $\{v\} \cup K$ ao seu pai $\{v'\} \cup K'$ é rotulada com o índice do vértice em $K'$ (visualizando-o como um conjunto ordenado) que não aparece em $K$. Quando o pai é $R$ a aresta é rotulada $\epsilon$.
  \end{enumerate}

  Note que a existência de um único vértice em $K' \setminus K'$ é garantida pela definição \ref{def:skeleton}. De fato, $v'$ precisa aparecer em $K$, caso contrário $K' = K$ e o pai de $\{v'\} \cup K'$ contém $K$. Isso contradiz o fato de que cada vértice em $S(T_k, R)$ é ligado à distância mínima da raiz.
\end{definition}

