\chapter{Fundamentos}
\label{cap:fundamentos}

Neste capítulo, apresentamos algumas definições que o leitor deve conhecer para compreender o trabalho.

Partimos do pressuposto de que o leitor conhece notações básicas de conjuntos.

\section{Grafos}

\begin{definition}[grafo]
  \cite{defgrafo}
  Um grafo é um par ordenado $G = (V, E)$. Os elementos de $V$ são chamados de vértices de $G$. Os elementos de $E$ são chamados de arestas de $G$ e consistem em pares (não-ordenados) de vértices. Dados $u, v \in V$, se $(u, v) \in E$ dizemos que $u$ e $v$ são adjacentes em $G$.
\end{definition}

\begin{definition}[grafo completo]
  \cite{defgrafocompleto}
  Um grafo $G = (V, E)$ é dito completo se $(u, v) \in E$ para todo $u, v \in V, u \neq v$.
\end{definition}

\begin{definition}[subgrafo induzido]
  \cite{defsubgrafo}
  Dado um grafo $G = (V, E)$ e um subconjunto $V'$ de $V$, o subgrafo de $G$ induzido por $V'$, $G' = (V', E')$, é o grafo formado pelos vértices $V' \subseteq V$ e arestas que só contém elementos de $V'$, ou seja, $E' = \{(u, v) \in E | u, v \in V'\}$.
\end{definition}

\begin{definition}[caminho]
  \cite{defcaminho}
  Dado um grafo $G = (V, E)$, um caminho é uma sequência de arestas que conectam uma sequência de vértices adjacentes distintos.
\end{definition}

\begin{definition}[distância]
  \cite{defdistancia}
  Dado um grafo $G = (V, E)$ e dois vértices $(u, v) \in V$, a distância entre $u$ e $v$ é o número de arestas num menor caminho que os conecte.
\end{definition}

\begin{definition}[árvore]
  \cite{defarvore}
  Dado um grafo $G = (V, E)$, dizemos que ele é uma árvore se cada dois vértices $u, v \in V$ são conectados por exatamente um caminho.
\end{definition}

\begin{definition}[$k$-clique]
  \label{def:kclique}
  \cite{defkclique} Seja $G = (V, E)$ um grafo. Um $k$-clique é um subconjunto dos vértices, $C \subseteq V$, tal que $(u, v) \in E \ \forall \ u, v \in C, u \neq v$ (ou seja, tal que o subgrafo induzido por $C$ é completo).
\end{definition}

\begin{definition}[$k$-tree e $k$-tree enraizada]
  \label{def:ktree}
  \cite{harary} Uma $k$-tree é definida da seguinte forma recursiva:

  \begin{enumerate}
    \item Um grafo induzido por um $k$-clique é uma $k$-tree.
    \item Se $T_k' = (V, E)$ é uma $k$-tree, $K \subseteq V$ é um $k$-clique e $v \not \in V$, então $T_k = (V \cup \{v\}, E \cup \{(v,x) \ | \  x \in K\})$ é uma $k$-tree.
  \end{enumerate}

  Uma $k$-tree enraizada é uma $k$-tree com um $k$-clique destacado $R = \{r_1, r_2, \cdots, r_k\}$ que é chamado de \emph{raiz} da $k$-tree enraizada.
\end{definition}

\section{Redes bayesianas}

% TODO
