\documentclass[a4paper,12pt]{book}

\usepackage[utf8]{inputenc}
\usepackage[brazil]{babel}
\usepackage{emptypage}
\usepackage{hyperref}
\usepackage{natbib}
\usepackage{amsmath,amssymb,amsfonts,amsthm}

\newtheoremstyle{definition}
{\topsep}{\topsep}
{}{}
{\bfseries}{}
{ }
{\thmname{#1}~\thmnumber{#2}\thmnote{ (#3)}.}

\theoremstyle{definition}
\newtheorem{definition}{Definição}

\newtheoremstyle{algorithm}
{\topsep}{\topsep}
{}{}
{\scshape}{}
{ }
{\thmnote{#3}\\}

\theoremstyle{algorithm}
\newtheorem{algorithm}{Algoritmo}

\newtheoremstyle{step}
{\topsep}{\topsep}
{}{}
{\itshape}{}
{1em}
{\thmname{#1}~\thmnumber{#2}.}

\theoremstyle{step}
\newtheorem{step}{Passo}

\frenchspacing
\linespread{1.5}

\title{Geração uniforme de \emph{k-trees} para aprendizado de redes bayesianas}
\author{Tiago Madeira}

\begin{document}

\frontmatter

  \thispagestyle{empty}
  \begin{center}
  \vspace*{1cm}
  Universidade de São Paulo\\
  Instituto de Matemática e Estatística\\
  Bachalerado em Ciência da Computação

  \vspace*{3cm}
  {\Large Tiago Madeira}

  \vspace{3cm}
  {
    \Large \bfseries
    Geração uniforme de \emph{k-trees} para aprendizado de redes bayesianas
  }

  \vspace{3cm}
  Supervisor: Prof. Dr. Denis Deratani Mauá

  \vspace{2cm}
  São Paulo

  Novembro de 2016
\end{center}

  \cleardoublepage\thispagestyle{empty}

  \pagenumbering{roman}
  \chapter*{Resumo}

O resumo ainda não foi escrito. % TODO

\vspace{1em}

\noindent \textbf{Palavras-chave:} sem, resumo, por, enquanto.


  \chapter*{Abstract}

The abstract has not been written yet. % TODO

\vspace{1em}

\noindent \textbf{Keywords:} no, abstract, yet.


  \tableofcontents
  \cleardoublepage

\mainmatter

  \chapter{Introdução}
\label{cap:introducao}

Em teoria dos grafos, \emph{$k$-trees} são consideradas uma generalização de árvores. Há interesse considerável em desenvolver ferramentas eficientes para manipular essa classe de grafos, porque todo grafo com \emph{treewidth} $k$ é um subgrafo de uma \emph{$k$-tree} e muitos problemas NP-completos podem ser resolvidos em tempo polinomial quando restritos a grafos com \emph{treewidth} limitada.

Com efeito, o artigo de Arnborg e Proskurowski \cite{arnborg} apresenta algoritmos para resolver em tempo linear problemas como, dado um grafo com \emph{treewidth} limitada:

\begin{itemize}
  \item Encontrar o tamanho máximo dos seus conjuntos independentes;
  \item Computar o tamanho mínimo dos seus conjuntos dominantes;
  \item Calcular seu número cromático; e
  \item Determinar se ele tem um ciclo hamiltoniano.
\end{itemize}

O problema que desperta nosso interesse em \emph{$k$-trees} é a inferência em redes bayesianas.

Uma rede bayesiana é um modelo probabilístico em grafo usado para raciocinar e tomar decisões em situações com incerteza através de técnicas de inteligência artificial e aprendizagem computacional. Ela representa uma distribuição de probabilidade multivariada num DAG (grafo acíclico dirigido) no qual os vértices correspondem às variáveis aleatórias do domínio e as arestas correspondem, intuitivamente, a influência de um vértice sobre outro.

Segundo Koller e Friedman \cite{koller}, a inferência em redes bayesianas em geral é NP-difícil; porém, se seu DAG possui \emph{treewidth} limitado, a inferência pode ser realizada em tempo polinomial. Daí a importância de aprender redes bayesianas que tenham \emph{treewidth} limitada.

A partir dessa motivação, este trabalho de conclusão de curso consistiu em estudar os conceitos de teoria dos grafos relacionados a \emph{$k$-trees} e implementar um algoritmo para gerar \emph{$k$-trees} de forma uniforme que possam ser usadas no aprendizado de redes bayesianas.

\vspace{2em}

A continuar: citar algoritmos usados, falar do artigo de Caminiti, justificar a escolha de Go para implementações, apresentar a organização da monografia. % TODO

  \cleardoublepage

  \chapter{Fundamentos}
\label{cap:fundamentos}

Neste capítulo, apresentamos definições fundamentais de teoria dos grafos, teoria da probabilidade e redes bayesianas que o leitor deve conhecer para compreender o trabalho.

Outras definições mais específicas, como as utilizadas para construir o algoritmo para codificar e decodificar \emph{$k$-trees} estão localizadas nos capítulos subsequentes.

Partimos do pressuposto de que o leitor conhece notações básicas de conjuntos.

\section{Grafos}

Nesta seção apresentamos de forma breve apenas os conceitos de teoria dos grafos necessários para a compreensão deste trabalho. Mais detalhes podem ser encontrados no livro de Bondy e Murty\cite{bondy}, que foi utilizado como referência.

\begin{definition}[grafo]
  Um grafo é um par ordenado $G = (V, E)$. Os elementos de $V$ são chamados de vértices de $G$. Os elementos de $E$ são chamados de arestas de $G$ e consistem em pares (não-ordenados) de vértices distintos\footnote{A rigor, por causa da palavra ``distintos'', essa é a definição do que a literatura costuma chamar de \emph{grafo simples}. Tal definição é utilizada porque neste trabalho não temos interesse em grafos que possuam arestas $(u, v)$ com $u=v$.}. Dados $u, v \in V$, se $(u, v) \in E$ dizemos que $u$ e $v$ são adjacentes em $G$.
\end{definition}

\begin{definition}[grafo dirigido]
  Um grafo $G = (V, E)$ é dito dirigido se $E$ consiste em pares \emph{ordenados} de vértices.
\end{definition}

\begin{definition}[grafo completo]
  Um grafo $G = (V, E)$ é dito completo se $(u, v) \in E$ para todo $u, v \in V, u \neq v$.
\end{definition}

\begin{definition}[subgrafo]
  Um grafo $F = (V_F, E_F)$ é chamado de subgrafo de $G = (V_G, E_G)$ se $V_F \subseteq V_G$ e $E_F \subseteq E_G$.
\end{definition}

\begin{definition}[subgrafo induzido]
  Dado um grafo $G = (V, E)$ e um subconjunto $V'$ de $V$, o subgrafo de $G$ induzido por $V'$, $G' = (V', E')$, é o grafo formado pelos vértices $V' \subseteq V$ e arestas que só contém elementos de $V'$, ou seja, $E' = \{(u, v) \in E \ | \  u, v \in V'\}$.
\end{definition}

\begin{definition}[caminho]
  Dado um grafo $G = (V, E)$, um caminho em $G$ é um subgrafo de $G$ cujos vértices podem ser arranjados numa sequência linear de forma que dois vértices são adjacentes se eles são consecutivos na sequência e não-adjacentes caso contrário. Se $u, v \in V$ pertencem a um caminho $P$, dizemos que eles estão conectados pelo caminho $P$.
\end{definition}

\begin{definition}[distância]
  Dado um grafo $G = (V, E)$ e dois vértices $(u, v) \in V$, a distância entre $u$ e $v$ é o número de arestas num menor caminho que os conecte.
\end{definition}

\begin{definition}[ciclo]
  Dado um grafo $G = (V, E)$, um ciclo em $G$ é um subgrafo de $G$ cujos vértices podem ser arranjados numa sequência cíclica de forma que dois vértices são adjacentes se eles são consecutivos na sequência e não-adjacentes caso contrário.
\end{definition}

\begin{definition}[DAG]
  Um grafo $G = (V, E)$ é chamado de DAG (do inglês \emph{directed acyclic graph}: grafo dirigido acíclico) se ele é dirigido e não possui ciclos.
\end{definition}

\begin{definition}[árvore]
  Dado um grafo $G = (V, E)$, dizemos que ele é uma árvore se cada dois vértices $u, v \in V$ são conectados por exatamente um caminho.
\end{definition}

\begin{definition}[$k$-clique]
  Seja $G = (V, E)$ um grafo. Um $k$-clique é um subconjunto dos vértices, $C \subseteq V$, tal que $(u, v) \in E \ \forall \ u, v \in C, u \neq v$ (ou seja, tal que o subgrafo induzido por $C$ é completo).
\end{definition}

\subsection{\emph{$k$-trees}}

\begin{definition}[\emph{$k$-tree}]
  \label{def:ktree}
  \cite{harary} Uma \emph{$k$-tree} é definida da seguinte forma recursiva:

  \begin{enumerate}
    \item Um grafo induzido por um $k$-clique é uma \emph{$k$-tree}.
    \item Se $T_k' = (V, E)$ é uma \emph{$k$-tree}, $K \subseteq V$ é um $k$-clique e $v \not \in V$, então $T_k = (V \cup \{v\}, E \cup \{(v,x) \ | \  x \in K\})$ é uma \emph{$k$-tree}.
  \end{enumerate}
\end{definition}

\begin{definition}[\emph{$k$-tree} enraizada]
  \cite{caminiti} Uma \emph{$k$-tree} enraizada é uma \emph{$k$-tree} com um $k$-clique destacado $R = \{r_1, r_2, \cdots, r_k\}$ que é chamado de \emph{raiz} da \emph{$k$-tree} enraizada.

  Na figura \ref{fig:rootedktree}(a), um exemplo de uma \emph{$k$-tree} com $k = 3$ e $n = 11$ vértices rotulados com inteiros em $[1, 11]$. Na figura \ref{fig:rootedktree}(b), a mesma \emph{$k$-tree}, dessa vez enraizada no clique $R = \{2, 3, 9\}$.

  \begin{figure}
    \begin{minipage}{0.5\textwidth}
      \centering
      \begin{tikzpicture}
          [scale=.6,auto=left,every node/.style={draw, circle, inner sep = 0pt, minimum width = 0.72cm}]
        \node (n10) at (1,9) {10};
        \node (n2) at (2.5,7) {2};
        \node (n1) at (1,4) {1};
        \node (n5) at (3,2.75) {5};
        \node (n7) at (2,1) {7};
        \node (n9) at (4.5,9.5) {9};
        \node (n6) at (4,5) {6};
        \node (n4) at (6,10.5) {4};
        \node (n3) at (8.5,6.5) {3};
        \node (n8) at (8,4.5) {8};
        \node (n11) at (9,9) {11};

        \foreach \from/\to in {n1/n2, n1/n5, n1/n7, n1/n8, n2/n3, n2/n5, n2/n6, n2/n8, n2/n9, n2/n10, n2/n11, n3/n4, n3/n5, n3/n8, n3/n9, n3/n10, n3/n11, n4/n9, n4/n11, n5/n7, n5/n8, n6/n8, n6/n9, n7/n8, n8/n9, n9/n10, n9/n11}
          \draw (\from) edge (\to);
      \end{tikzpicture}

      (a)
    \end{minipage}\begin{minipage}{0.5\textwidth}
      \centering
      \begin{tikzpicture}
          [scale=.6,auto=left,every node/.style={draw, circle, inner sep = 0pt, minimum width = 0.72cm}]
        \node (n10) at (1,8.25) {10};
        \node[fill=gray!30] (n2) at (1.5,10.5) {2};
        \node (n1) at (2,3.75) {1};
        \node (n5) at (3,6) {5};
        \node (n7) at (3.5,1.5) {7};
        \node[fill=gray!30] (n9) at (4.5,10.5) {9};
        \node (n6) at (6,6) {6};
        \node (n4) at (9,4) {4};
        \node[fill=gray!30] (n3) at (7.5,10.5) {3};
        \node (n8) at (4.5,8.25) {8};
        \node (n11) at (8,8.25) {11};

        \foreach \from/\to in {n1/n5, n1/n7, n2/n5, n2/n8, n2/n9, n2/n10, n2/n11, n3/n8, n3/n9, n3/n10, n3/n11, n4/n9, n4/n11, n5/n7, n5/n8, n6/n8, n6/n9, n8/n9, n9/n10, n9/n11}
          \draw (\from) edge (\to);

        \draw (n1) edge [bend right=20] (n8);
        \draw (n1) edge [bend left=50] (n2);
        \draw (n2) edge [bend left] (n3);
        \draw (n2) edge [bend right=20] (n6);
        \draw (n3) edge [bend left] (n4);
        \draw (n3) edge [bend left=20] (n5);
        \draw (n7) edge [bend right=20] (n8);
      \end{tikzpicture}

      (b)
    \end{minipage}

    \caption{
      \textbf{(a)} Uma \emph{$3$-tree} $T_3$ com 11 vértices.
      \textbf{(b)} A mesma \emph{$3$-tree} ($T_3$) enraizada no clique $\{2, 3, 9\}$.
    }
    \label{fig:rootedktree}
  \end{figure}
\end{definition}

\begin{definition}[\emph{partial $k$-tree}]
  \cite{bodlaender} Um subgrafo de uma \emph{$k$-tree} é chamado de \emph{partial $k$-tree}. Um grafo é uma \emph{partial $k$-tree} se e só se ele tem \emph{treewidth} menor ou igual a $k$.
\end{definition}

\section{Probabilidade}

A escrever. \cite{koller} % TODO

\section{Redes bayesianas}

A escrever. \cite{koller} % TODO

% TODO: onde falar de aprendizagem e inferência?

  \cleardoublepage

  \chapter{Geração aleatória de \emph{$k$-trees}}
\label{cap:geracao}

O problema de gerar \emph{$k$-trees} está intimamente relacionado ao problema de codificá-las e decodificá-las. De fato, se há uma codificação bijetiva que associa \emph{$k$-trees} a \emph{strings}, basta gerar \emph{strings} aleatórias para gerar \emph{$k$-trees} aleatórias.

Neste capítulo, apresentamos o problema de codificar \emph{$k$-trees}, discutimos a solução linear para codificar e decodificar \emph{$k$-trees} de forma bijetiva proposta por Caminiti et al\cite{caminiti}, explicamos como ela foi implementada neste trabalho para gerar \emph{$k$-trees} aleatórias e mostramos os resultados obtidos.

\section{Introdução à codificação de \emph{$k$-trees}}

O problema de codificar árvores já foi amplamente estudado na literatura. Como destaca Caminiti et al\cite{caminiti}:

\begin{quotation}
  Codificar árvores rotuladas por meio de \emph{strings} de rótulos de vértices é uma alternativa interessante à representação usual de estruturas de dados de árvore na memória e tem muitas aplicações práticas (por exemplo, algoritmos evolucionários sobre árvores, geração aleatória de árvores, compressão de dados e computação do volume de floresta de grafos). Diversos códigos bijetivos diferentes que realizam associações entre árvores rotuladas e \emph{strings} de rótulos foram introduzidas. De um ponto de vista algorítmico, o problema foi cuidadosamente investigado e algoritmos ótimos de codificação e decodificação desses códigos são conhecidos.
\end{quotation}

Em 1889, Cayley\cite{cayley} demonstrou que para um conjunto de $n$ vértices distintos existem $n^{n-2}$ árvores possíveis. Desde lá, foram criados vários códigos para associar \emph{strings} e árvores.

Um dos mais conhecidos é o código de Prüfer\cite{prufer}, que surgiu em 1918 e é bijetivo, associando cada árvore (rotulada) de $n$ vértices a uma lista distinta de comprimento $n-2$ no alfabeto dos rótulos da árvore.

\emph{$k$-trees}\cite{harary} são consideradas uma generalização de árvores. Há interesse considerável em desenvolver ferramentas eficientes para manipular essa classe de grafos, porque todo grafo com \emph{treewidth} $k$ é um subgrafo de uma \emph{$k$-tree} e muitos problemas NP-completos podem ser resolvidos em tempo polinomial quando restritos a grafos com \emph{treewidth} limitada, como destacado na \textbf{Introdução} deste trabalho.

Há estudos sobre a codificação de \emph{$k$-trees} há pelo menos quatro décadas. Em 1970, Rényi e Renýi apresentaram uma codificação redundante (ou seja, não bijetiva) para um subconjunto de \emph{$k$-trees} rotuladas que chamamos de \emph{$k$-trees} de Rényi e que são definidas como segue:

\begin{definition}[\emph{$k$-tree} de Rényi]
  \cite{renyi} Uma \emph{$k$-tree} de Rényi $R_k$ é uma \emph{$k$-tree} enraizada com $n$ vértices rotulados em $[1, n]$ e raiz $R = \{n-k+1, n-k+2, \cdots, n\}$.
\end{definition}

Entretanto, até onde sabemos, apenas em 2008 surgiu um código bijetivo para \emph{$k$-trees} com algoritmos lineares de codificação e decodificação. Foram esses algoritmos, propostos por Caminiti et al\cite{caminiti}, que implementamos neste trabalho.

\section{A solução de Caminiti et al}

O artigo \emph{``Bijective Linear Time Coding and Decoding for $k$-Trees''}\cite{caminiti} apresenta um código bijetivo para \emph{$k$-trees} rotuladas, juntamente a algoritmos lineares para realizar a codificação e a decodificação.

O código é formado por uma permutação de tamanho $k$ e uma generalização do \emph{Dandelion Code}\cite{egecioglu}, que consiste em $n-k-2$ pares (onde $n$ é o número de vértices) definidos no conjunto $\{ ( 0, \varepsilon ) \} \cup ([1,n-k] \times [1,k])$. Portanto, dizemos que a codificação das \emph{$k$-trees} associa elementos em $\mathcal{T}^n_k$ (conjunto das \emph{$k$-trees} com $n$ vértices) com elementos em:

$$
\mathcal{A}^n_k = { [1,n] \choose k } \times (\{ ( 0, \varepsilon ) \} \cup ([1,n-k] \times [1,k]))^{n-k-2}
$$

Os algoritmos consistem em uma série de transformações. Para compreendê-los, é necessário definir esqueleto de uma \emph{$k$-tree} enraizada e árvore característica:

\begin{definition}[esqueleto de uma \emph{$k$-tree} enraizada]
  \label{def:skeleton}
  \cite{caminiti} O esqueleto de uma \emph{$k$-tree} enraizada $T_k$ com raiz $R$, denotado por $S(T_k, R)$, é definido da seguinte forma recursiva:

  \begin{enumerate}
    \item Se $T_k$ é apenas o $k$-clique $R$, seu esqueleto é uma árvore com um único vértice $R$.
    \item Dada uma \emph{$k$-tree} enraizada $T_k$ com raiz $R$, obtida por $T_k'$ enraizada em $R$ através da adição de um novo vértice $v$ conectado a um $k$-clique $K$ (ver definição \ref{def:ktree}), seu esqueleto $S(T_k, R)$ é obtido adicionando a $S(T_k', R)$ um novo vértice $X = \{v\} \cup K$ e uma nova aresta $(X, Y)$, onde $Y$ é o vértice de $S(T_k', R)$ que contém $K$ com uma distância mínima da raiz. Chamamos $Y$ de pai de $X$.
  \end{enumerate}
\end{definition}

\begin{definition}[árvore característica]
  \cite{caminiti} A árvore característica $T(T_k, R)$ de uma \emph{$k$-tree} enraizada $T_k$ com raiz $R$ é obtida rotulando os vértices e arestas de $S(T_k, R)$ da seguinte forma:

  \begin{enumerate}
    \item O vértice $R$ é rotulado $0$ e cada vértice $\{v\} \cup K$ é rotulado $v$;
    \item Cada aresta do vértice $\{v\} \cup K$ ao seu pai $\{v'\} \cup K'$ é rotulada com o índice do vértice em $K'$ (visualizando-o como um conjunto ordenado) que não aparece em $K$. Quando o pai é $R$ a aresta é rotulada $\varepsilon$.
  \end{enumerate}

  Note que a existência de um único vértice em $K' \setminus K$ é garantida pela definição \ref{def:skeleton}. De fato, $v'$ precisa aparecer em $K$, caso contrário $K' = K$ e o pai de $\{v'\} \cup K'$ contém $K$. Isso contradiz o fato de que cada vértice em $S(T_k, R)$ é ligado à distância mínima da raiz.
\end{definition}

\subsection{Codificação}

O algoritmo para codificar uma \emph{$k$-tree} rotulada consiste em seis passos. Aqui apresentamos esse algoritmo detalhando nossa implementação.

\begin{algorithm}[Algoritmo de codificação]
  \textbf{Entrada:} uma \emph{$k$-tree} $T_k$ com $n$ vértices\\
  \textbf{Saída:} um código em $\mathcal{A}^n_k$

  \begin{enumerate}
    \item Identificar $Q$, o $k$-clique adjacente à folha de maior rótulo $l_M$ de $T_k$;
    \item Através de um processo de re-rotulação $\phi$ (computado a partir de $Q$ e definido a seguir), transformar $T_k$ numa \emph{$k$-tree} de Rényi $R_k$;
    \item Gerar a árvore característica $T$ para $R_k$;
    \item Computar o \emph{Dandelion Code} generalizado $S$ para $T$;
    \item Remover da \emph{string} obtida $S$ o par correspondente a $\phi(l_M)$;
    \item Retornar o código $(Q, S) \in \mathcal{A}^n_k$.
  \end{enumerate}

  Na nossa implementação, uma \emph{$k$-tree} (estrutura definida no pacote {\tt ktree}) é representada através de uma lista de adjacências ({\tt Adj}) e um inteiro $k$ ({\tt K}). % TODO: é preciso definir lista de adjacência (em Fundamentos?).

  O algoritmo de codificação é implementado pela função {\tt CodingAlgorithm} do pacote {\tt codec}. A seguir, detalhamos os seis passos.

  \begin{step}
    A escrever. % TODO
  \end{step}

  A escrever. % TODO
\end{algorithm}

\subsection{Decodificação}

A escrever. % TODO

\section{Experimentos e resultados}

A escrever. % TODO


  \cleardoublepage

  \chapter{Aprendizado de redes bayesianas}
\label{cap:aprendizado}

A ser escrito. % TODO

  \cleardoublepage

  \chapter{Conclusão}
\label{cap:conclusao}

Ainda não foi escrita. % TODO

  \cleardoublepage

\backmatter

  \bibliographystyle{plain}
  \bibliography{referencias}

\end{document}
